\documentclass[12pt,a4paper]{article}
\usepackage[margin=2cm]{geometry}

\usepackage{titlesec}

\renewcommand \thesubsection {\arabic{subsection}}
\titleformat{\subsection}{\normalfont}{\arabic{subsection}.}{1em}{}
\titlespacing{\subsection}{1em}{\baselineskip}{\baselineskip}
\titleformat{\subsubsection}{\normalfont}{(\alph{subsubsection})}{1em}{}
\titlespacing{\subsubsection}{2em}{\baselineskip}{\baselineskip}

\usepackage{graphicx}
\usepackage{listings}
\lstset{
	basicstyle=\ttfamily,
}
\usepackage{tikz}
\usetikzlibrary{calc}
\usepackage[hyphens]{url}
\usepackage{hyperref}
\usepackage{algorithm}
\usepackage{algpseudocode}
\usepackage{amsmath, amssymb}

\usepackage{xeCJK}
\setCJKmainfont{Noto Sans CJK TC}

\title{Final Project: FHE-secured CAPTCHA Solver}
\author{蕭凱鴻 B13902046, 蔡兆豐 B13902110}
\date{}

\begin{document}

\maketitle
% \tableofcontents

\subsection{目標}

模擬一個場景,使用者的 CAPTCHA 輸入會加密,伺服器對加密輸入與正確答案進行匹配驗證,不需看到明文。
挑戰:如何在 FHE 下做字元等值比較。

\subsection{實作架構}

我們全部都使用 python 實作,搭配 zama 官方提供的 concrete API 來達到 FHE 加密。程式分為 4 個檔案:\verb|app2.py|、\verb|client.py|、\verb|server.py|、\verb|service.py|。\\

啟動 \verb|python app2.py| 以後會聯絡 \verb|server.py|。server 會生成隨機的 6 個字元字串並編譯好用來加密比較兩個字串的 circuit,把 circuit 的計算部份和加密後的正確答案交給 \verb|service.py|。然後 app 會在 \verb|127.0.0.1:5000| 啟動 flask 框架的網頁,顯示 captcha 圖片給使用者。\\

使用者輸入完以後,會由 \verb|client.py| 加密完把東西傳給 \verb|service.py| 做運算,和正確答案做逐字比較,要每個字元的 FHE 加密都和答案一樣才算成功。比較完成以後會把結果回傳給 client 和 server,就能進到認證成功或失敗的頁面。

\subsection{一些問題}

關於這樣的架構,由於 zama 並沒有在 python 支援編譯器產生的 public key 傳送,要的話只能用 rust 寫。

\subsection{Github}

\url{https://github.com/GrandTiger1729/BDA-FHE.git}

\end{document}
